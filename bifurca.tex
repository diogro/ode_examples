\documentclass{beamer}
%\documentclass[handout]{beamer}
%\documentclass{article}
%\usepackage{beamerarticle}

\mode<presentation>
{
    %\usetheme{Warsaw}
    %\usetheme{Frankfurt}
    \usetheme[subsection=false]{Dresden}
    %\usecolortheme{dove}
    \usecolortheme{beaver}
    \useinnertheme{circles}

    \setbeamercovered{dynamic}
    % If you wish to uncover everything in a step-wise fashion, uncomment
    % the following command: 
    %\beamerdefaultoverlayspecification{<+->}
    %\pgfdeclareimage[height=0.6cm]{university-logo}{IFT.jpg}
    %\logo{\pgfuseimage{university-logo}}
}

\usepackage[english]{babel}
\usepackage[utf8]{inputenc}

\usepackage[T1]{fontenc}
\usepackage{lmodern}
\usepackage{graphicx}


\title[Bifurcation diagrams]{Tutorial on Qualitative analysis and Bifurcation diagrams}

\author[Renato Coutinho]{Renato Mendes Coutinho (IFT - Unesp)}

\date[SSSMB 2014] % (optional, should be abbreviation of conference name)
{
    {\footnotesize \texttt{renatomc@ift.unesp.br}}\\
\vspace{0.5cm}S\~ao Paulo\\February, 2014}

\begin{document}

\begin{frame}
    \titlepage
\end{frame}

\begin{frame}{}
    \tableofcontents
\end{frame}

\section{Introduction}

\begin{frame}{I have a model! What now?}
    \pause
    \begin{block}{Principle I}
        I have equations, so I want solutions!
    \end{block}
    \pause
    \vfill
    We will use as example a nice and simple model that we all should care
    about: the next step in predator--prey (or consumer--resource) models: the
    \textbf{Rosenzweig--MacArthur} model.
\end{frame}

\begin{frame}{The Rosenzweig--MacArthur model}
    \begin{itemize}
        \item Includes the same ingredients as the Lotka-Volterra equations,
        \item plus a carrying capacity for the resource,
        \item and plus a saturation in the predation rate.
    \end{itemize}
    \pause
    \begin{align*}
        \frac{dR}{dt} &= rR \left( 1 - \frac{R}{K} \right) - \frac{a R C}{1+ahR} \\
        \frac{dC}{dt} &= \frac{e a R C}{1+ahR} - d C
    \end{align*}
\end{frame}

\begin{frame}{Rosenzweig--MacArthur model solutions}
    \begin{itemize}
        \item By Principle I, we want the solution of the model.
        \item We resort to numerical integration, but then\ldots which
            parameters should I use? And which initial conditions?
        \item For now, let's just guess some parameters, and pick a few
            different initial conditions.
        \item
            \includegraphics[height=1.5ex]{ipynblogo.png}
            \pause
        \item It seems that initial conditions don't matter for the final
            long-term solution: a fixed point.
    \end{itemize}
\end{frame}

\section{``Visual'' qualitative analysis}

\begin{frame}{The phase space flow and the fixed point}
    \begin{itemize}
        \item The differential equations define a flow in the phase space: at
            each point, there's a direction the solution must follow when it
            goes through that point.
        \item
            \includegraphics[height=1.5ex]{ipynblogo.png}
            \pause
        \item The size (the magnitude) of the arrows become small near
            the fixed point.
        \item That is, at the fixed point, $\frac{dC}{dt} = 0$ and
            $\frac{dR}{dt} = 0$.
        \item The flow spirals towards the center, so any initial condition
            approaches the center.
    \end{itemize}
\end{frame}

\begin{frame}{Messing a little with the parameters\ldots}
    \begin{itemize}
        \item We have seen that predator-prey systems tend to oscillate, but in this
            case, the long-term solution is stationary.
        \item Let's, for example, increase the carrying capacity $K$ a little.
        \item
            \includegraphics[height=1.5ex]{ipynblogo.png}
            \pause
        \item Now the arrows inside spirals outward, and the flow outside
            spirals inward, towards a \textbf{limit cycle}.
        \item The fixed point is still there (the arrows' sizes go to zero
            in the center), but now the solution moves away from it.
        \item We say that the fixed point became \textbf{unstable}.
        \item A change in the stability of a fixed point is a
            \textbf{bifurcation}.
    \end{itemize}
\end{frame}

\begin{frame}{Varying a parameter systematically: the bifurcation diagram}
    \begin{itemize}
        \item Changing the values of parameters haphazardly, it may be hard to
            see and to synthesize what happens in the system.
        \item Let's imagine you change the value of a parameter by a very
            small value:
        \pause
        \item the expectation is that the solution changes only a tiny bit.
        \item But if we sweep a range of values in small steps, we will see a
            parameter value where the solution attains a new behavior: the bifurcation
            point.
        \item Let's do this increasing the resource carrying capacity $K$:
            what do you expect it is going to happen?
        \item
            \visible<2>{\includegraphics[height=1.5ex]{ipynblogo.png}}
    \end{itemize}
\end{frame}

\begin{frame}{The paradox of enrichment}
    \begin{itemize}
        \item For very small $K$, the predator is extinct,
        \item for intermediate $K$, the solution goes to a fixed point (the
            minimum and maximum of the solution have the same value!)
        \item and for high $K$, there are oscillations with amplitudes that
            increase with $K$.
            \pause
        \item The ``paradox of enrichment'' means that boosting
            the resource population can lead to extinction either of the
            resource or of the consumer (or both), because the solution passes
            closer and closer to zero.
    \end{itemize}
\end{frame}

\section{Systems that depend explicitly on time}

\begin{frame}{Seasonal consumer resource dynamics}
    \begin{itemize}
        \item In certain situations, we may want to include explicit temporal
            dependence into our models.
        \item Seasonality, environmental fluctuations and experimental
            manipulation are some clear reasons why you would need that.
        \item Let's see what happens when we take our Rosenzweig-MacArthur
            model and introduce a seasonal growth rate $r = r_0 (1+\alpha
            \sin(2\pi t/T))$ 
        \item We are going to make a small perturbation ($\alpha$ small), so
            we shouldn't see much happening.
        \item
            \includegraphics[height=1.5ex]{ipynblogo.png}
    \end{itemize}
\end{frame}

\begin{frame}{A resonance diagram}
    \begin{itemize}
        \item The population oscillates together with the seasonal variation,
            even though the system with $K=10$ didn't oscillate -- but the
            amplitude is small.
        \item We now do a kind of bifurcation diagram, but now what we vary is
            the \textbf{frequency of the external oscillations}.
        \item
            \includegraphics[height=1.5ex]{ipynblogo.png}
            \pause
        \item There's a sharp peak in the amplitude around a certain
            frequency. This is called a \textbf{resonance}.
        \item If we go back to the first plot (with $K=10$ without seasonal
            fluctuations), we find that the period of the oscillations in the
            transient is around 23.
        \item No, not a coincidence\ldots
    \end{itemize}
\end{frame}

\section{A few comments}

\begin{frame}{What if there are more than 2 equations?}
    \begin{itemize}
        \item In that case, the phase space has more than 2 dimensions and
            doesn't fit into a nice 2-d plot.
        \item You can still try to plot planes of the phase space, specially
            ones containing the fixed points of interest.
        \item Bifurcation diagrams are still very much useful: you don't have
            to plot all the curves to characterize the solution.
    \end{itemize}
\end{frame}

\begin{frame}{What if I want to explore a 10-dimensional parameter
    space?}
    \begin{itemize}
        \item Avoid that: you can reduce the number of parameters
            rescaling your variables (adimensionalizing)
        \item You can also restrict the values considered based on data and on
            careful judgement: not every parameter has the same relevance to
            the outcomes.
    \end{itemize}
    \pause

    But I still have 10 parameters left!

    \begin{itemize}
        \item Well, Good luck!
        \item You will probably have to sample the space, rather than
            go through the whole thing. A recommended method is to use
            so-called
            \href{http://en.wikipedia.org/wiki/Latin_hypercube_sampling}{Latin Hypercube
            samples},
            that uses a random sampling while ensuring a roughly
            regularly-spaced distribution.
    \end{itemize}
\end{frame}

\begin{frame}{}
    Thanks for your attention!

    \vspace{4ex}
    All the code for the solutions and plots shown are available at 
    \href{http://ecologia.ib.usp.br/ssmb/doku.php?id=2014:courses:kraenkel:start}{the
    wiki of the course}
\end{frame}

\end{document}
